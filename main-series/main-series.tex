\documentclass[12pt]{book}

\usepackage{parskip}
\usepackage{xcolor}
\usepackage{Times}

\setlength{\parskip}{\baselineskip}%
\setlength{\parindent}{0pt}%

\pagecolor[rgb]{0.1, 0.1, 0.1}
\color[rgb]{1, 1, 1}

\title{Apocalyptic Invasion}
\author{Rayhan Wijaya}
\date{February 2023}

\begin{document}

\maketitle

\section*{Prologue}

She took the two kids to the park for a picnic, while
their parents were off working. Numerous other kids were
also at the park. Although SC could handle the pile of
tasks at home, she definitely needed some rest.

She lied on the grass to clear her mind once the two kids
were done eating their picnic meal. One of the kids she
serves swiftly ran towards SC, "Are we going to have
hotdogs for dinner?", the kid asked. SC attempted to
respond to the question, but then she remembered that she
was created with no voice chip in mind, and decided to
resort to body gestures, replying with a shake of her
head. The kid seemed disappointed, but then quickly
resumed playing with the other children.

SC took a glance at the other park-goers and noticed that
they all had pets, which the family she serves did not
have. She wondered how it would feel like to have an
animal companion in the household. How would the family
react to having a pet? Would it help her better
understand the humans she was programmed to assist? Would
it be a hassle to take care of? Would she enjoy the
company…? These were some of the question that was going
through SC’s mind.

While she was lost in her thoughts, SC noticed that the
same kid ran off with an especially costly snack from a
nearby hotdog stand, resulting in the seller becoming
furious. SC immediately got up and quickly intervened,
returning the hotdog to the seller.

However, the seller mocked SC for not watching the kid
closely enough. "I apologise if I’ve failed to watch the
kids attentively. I acted as soon as I possibly could to
return the stolen food", SC replied. "Though, the problem
has been resolved now. There’s no need to discuss it
further", SC continued before realising that all her
fluff talk was just in her head. Without a proper way to
communicate, she ended the argument with a simple nod.
Even though SC couldn’t speak due to not having a voice
chip, she still made an attempt to let the kid know,
using hand gestures, that it’s not okay to take things
without paying for them. The kid didn’t seem to pay
attention to her at all, leaving SC frustrated.

\clearpage

SC returned to lying on the grass, pondering that brief
argument. Not long after, SC brought the two kids back
home and quickly tucked them to nap. Telling them stories
before providing them their blankets. 

Once the kids were asleep, SC went outside, glancing at
the midnight stars on the porch. She could sense the wind
blowing, but she didn't really feel anything. She was not
designed to experience emotions or sensations in the way
humans do. She instead registers temperature, humidity,
wind speed, and adjusted her systems accordingly to
optimize her performance.

\section*{Chapter 1}

\section*{Chapter 2}

\end{document}
